\section{VHDL CODE}
In this chapter will be presented the main modules that compose the architecture of the \textbf{Perceptron with sigmoid activation function}.
\subsection{Modules List}
As presented in the last chapter, I have followed a similar approach for creating the architecture. The following modules were created:

\begin{itemize}
	\item Perceptron
	\begin{itemize}
		\item Parallel$\_$Multiplier
  		\begin{itemize}
  			\item Unsigned Parallel Multiplier
  			\begin{itemize}
	  			\item Full Adder
	  			\item Half Adder
  			\end{itemize}
		\end{itemize}
	\end{itemize}
	\begin{itemize}
		\item Tree$\_$Adder
		\begin{itemize}
			\item Ripple$\_$Carry$\_$Adder$\_$Pipelined
			\begin{itemize}
				\item DFF
				\item Full Adder
			\end{itemize} 
		\end{itemize} 
	\end{itemize}
	\begin{itemize}
		\item Sigmoid$\_$Lut$\_$2048
	\end{itemize}
\end{itemize}

A \textbf{bottom-up strategy} was followed in order to build up the architecture: starting from some modules that will made up the architecture, after finishing each of them some testbenches were written in order to test each building block of the \textbf{Perceptron} (See next chapter for details).
\subsection{Perceptron}
The main hardware description of the architecture. This module will connect all the other modules in order to create the correct architecture. In order to not show too much lines of code only the entity definition of this module will be shown.
\begin{lstlisting}[language=VHDL]
	entity Perceptron is
	port(
	
		-- x_1 to x_10 inputs of the perceptron with 8 bits
		x_1: in std_logic_vector(7 downto 0);
		x_2: in std_logic_vector(7 downto 0);
		x_3: in std_logic_vector(7 downto 0);
		x_4: in std_logic_vector(7 downto 0);
		x_5: in std_logic_vector(7 downto 0);
		x_6: in std_logic_vector(7 downto 0);
		x_7: in std_logic_vector(7 downto 0);
		x_8: in std_logic_vector(7 downto 0);
		x_9: in std_logic_vector(7 downto 0);
		x_10: in std_logic_vector(7 downto 0);
		
		-- w_1 to w_10 inputs of the perceptron with 9 bits
		w_1: in std_logic_vector(8 downto 0);
		w_2: in std_logic_vector(8 downto 0);
		w_3: in std_logic_vector(8 downto 0);
		w_4: in std_logic_vector(8 downto 0);
		w_5: in std_logic_vector(8 downto 0);
		w_6: in std_logic_vector(8 downto 0);
		w_7: in std_logic_vector(8 downto 0);
		w_8: in std_logic_vector(8 downto 0);
		w_9: in std_logic_vector(8 downto 0);
		w_10: in std_logic_vector(8 downto 0);
		
		-- b input of the perceptron with 9 bits
		b: in std_logic_vector(8 downto 0);
		
		clk: in std_logic;
		rst: in std_logic;
		
		-- output of the perceptron 16 bits
		f_z: out std_logic_vector(15 downto 0)
	);
	end Perceptron;
\end{lstlisting}

In the rest of this modules are instantiated and linked the various submodules that made up the \textbf{Perceptron} module.

\subsection{Parallel Multiplier}
This module has the duty to convert the inputs, which are signed with a 2's complement representation, link their unsigned representation with the \textbf{Unsigned Parallel Multiplier} module and then reconvert the product in the signed representation. The general architecture is shown in Figure 5.
\begin{lstlisting}[language=VHDL]
	entity Parallel_Multiplier is
		generic (
			Nbit_a : positive; 
			Nbit_b: positive
		);
		port(
			a_p_signed: in std_logic_vector(Nbit_a - 1 downto 0);
			b_p_signed: in std_logic_vector(Nbit_b - 1 downto 0);
			-- The product will need Nbit_a + Nbit_b bits
			p_signed: out std_logic_vector(Nbit_a + Nbit_b - 1 downto 0)
		);
	end entity Parallel_Multiplier;
	
	architecture rtl of Parallel_Multiplier is
	
		-- Building blocks of the Parallel Multiplier
		component Unsigned_Parallel_Multiplier
			generic(
				Nbit_a : positive;
				Nbit_b : positive
			);
			port(
				a_p: in  std_logic_vector(Nbit_a - 1 downto 0);
				b_p : in  std_logic_vector(Nbit_b - 1 downto 0);
				p   : out std_logic_vector(Nbit_a + Nbit_b - 1 downto 0)
			);
		end component Unsigned_Parallel_Multiplier;
		
		
		-- Unsigned component (will work for the unsigned parallel multiplier
		signal p_unsigned: std_logic_vector(Nbit_a + Nbit_b - 1 downto 0);
		signal a_p_unsigned: std_logic_vector(Nbit_a - 1 downto 0);
		signal b_p_unsigned: std_logic_vector(Nbit_b - 1 downto 0);
		
		-- will carry the sign bit for the signed rapresentation of the inputs
		signal a_sign: std_logic;
		signal b_sign: std_logic;
		
	begin
	
		-- Compute the unsigned representation from the signed one
		a_p_unsigned <= std_logic_vector(abs(signed(a_p_signed)));
		b_p_unsigned <= std_logic_vector(abs(signed(b_p_signed)));
		
		-- 2's complement rapresentation, the result sign uis computed through the xor op. between a and b
		p_signed <= std_logic_vector(unsigned(not(p_unsigned)) + 1) when (((a_sign xor b_sign) = '1')) else p_unsigned;
		
		-- Getting of the sign from a and b (the MSB of the C2 representation)
		a_sign <= a_p_signed(Nbit_a - 1);
		b_sign <= b_p_signed(Nbit_b - 1);
		
		unsigned_parallel_mul: Unsigned_Parallel_Multiplier
			generic map(
				Nbit_a => Nbit_a,
				Nbit_b => Nbit_b
			)
			port map(
				a_p =>  a_p_unsigned,
				b_p =>  b_p_unsigned,
				p   => p_unsigned
			);
		
	end architecture rtl; 
\end{lstlisting}

\subsection{Unsigned Parallel Multiplier}
This module will effectively compute a multiplication. Through the replication of the architecture shown in Figure 6 and 7 this module will return the \textit{unsigned product} of two \textit{unsigned operands}. In order to implement the architecture in the simplest way a \textbf{Structured approach} has been followed in the description. Some lines of code were just skipped in order to show only a general schema, for further details there is also the source code.
\begin{lstlisting}[language=VHDL]
	entity Unsigned_Parallel_Multiplier is
		generic (
			Nbit_a : positive; 
			Nbit_b: positive
		);
		port(
			-- Unsigned representation of inputs
			a_p: in std_logic_vector(Nbit_a - 1 downto 0);
			b_p: in std_logic_vector(Nbit_b - 1 downto 0);
			
			-- p = a_p * b_p
			p: out std_logic_vector(Nbit_a + Nbit_b - 1 downto 0)
		);
	end entity Unsigned_Parallel_Multiplier;
	
	architecture rtl of Unsigned_Parallel_Multiplier is
		-- Building blocks of the Unsigned Parallel Multiplier
		component FULL_ADDER is
			...
		end component;
		
		component HALF_ADDER is
			...
		end component;
		
		-- Will hold the carry signals among the whole architecture
		signal carry_signal: std_logic_vector((Nbit_a - 1)*(Nbit_b - 1) - 1 downto 0);
		signal last_carry_signal: std_logic_vector((Nbit_b - 1) downto 0);
		
		-- Will hold the sum result of the FA and HA among the whole architecture
		signal sum_signal: std_logic_vector((Nbit_a - 1)*(Nbit_b - 2) - 1 downto 0);  
		
		-- will hold the precomputed values for the inputs a and b of the various Half Adder and Full Adder
		signal a_multiplier: std_logic_vector(Nbit_a + Nbit_b - 2 downto 0);
		signal b_multiplier: std_logic_vector((Nbit_a - 1)*(Nbit_b - 1) - 1 downto 0);
	
	begin
		
		-- First bit of the result
		p(0) <= (a_p(0) and b_p(0));
		
		
		-- Computation of the various inputs of each HA and FA
		d_process: process(a_p, b_p)
		begin
		
		for j in 1 to Nbit_b loop
		a_multiplier(j - 1) <= (a_p(0) and b_p(Nbit_b - j));
		end loop;
		
		...
		end process d_process;
		
		
		-- Architecture will follow schema of the Parallel Multiplier
		-- Row index i
		GEN_a: for i in 1 to Nbit_a generate
			-- Column index j
			GEN_b: for j in 1 to Nbit_b - 1 generate 
				FIRST_ROW: if i=1 generate
				-- In the first Row only HA
				LEFT: if j < Nbit_b -1 generate
				ROW1_LEFT: HALF_ADDER
				port map
				(
				a    => a_multiplier(j - 1), 
				b    => b_multiplier(j - 1), 
				s    => sum_signal(j - 1), 
				cout => carry_signal(j - 1) 
				);
				end generate LEFT;
				RIGHT: if j = Nbit_b - 1 generate
				ROW1_RIGHT: HALF_ADDER
				port map
				(
				a    => a_multiplier(j - 1), 
				b    => b_multiplier(j - 1), 
				s    => p(1), -- Result bit 
				cout => carry_signal(j - 1) 
				);
				end generate RIGHT; 
				end generate FIRST_ROW;
				
				
				INTERNAL_ROW: if i > 1 and i < Nbit_a generate
					...
				end generate INTERNAL_ROW;
				
				
				LAST_ROW: if i = Nbit_a generate
					...
				end generate LAST_ROW;
			end generate GEN_b;
		end generate GEN_a;
	end architecture rtl; 
\end{lstlisting}

\subsection{Tree Adder}
This module will take up the ten multiplication results and the bias and will sum up every term by making the computation shown at the equation (1). Even in this case will not be shown the architectural code due to the fact that consist only in linking some submodules in the proper way in order to replicate the architecture shown in Figure (9).
\begin{lstlisting}[language=VHDL]
entity Tree_Adder is
	port(
		-- Inputs: result of the multiplication of xi*wi
		in_1: in std_logic_vector(16 downto 0);
		in_2: in std_logic_vector(16 downto 0);
		in_3: in std_logic_vector(16 downto 0);
		in_4: in std_logic_vector(16 downto 0);
		in_5: in std_logic_vector(16 downto 0);
		in_6: in std_logic_vector(16 downto 0);
		in_7: in std_logic_vector(16 downto 0);
		in_8: in std_logic_vector(16 downto 0);
		in_9: in std_logic_vector(16 downto 0);
		in_10: in std_logic_vector(16 downto 0);
		
		-- Bias input
		b: in std_logic_vector(8 downto 0);
		clk: in std_logic;
		rst: in std_logic;
		
		-- Output
		z: out std_logic_vector(20 downto 0)
	);
end Tree_Adder;

\end{lstlisting}

\subsubsection{Ripple Carry Adder Pipelined}
This module will be the main building block of the \textbf{Tree Adder} module. In order to reduce the logic chain some registers were added by exploiting the \textbf{DFF} module as seen in the Lab lectures.
\begin{lstlisting}[language=VHDL]
entity Ripple_Carry_Adder_Pipelined is
generic (Nbit: positive);
	port(
		-- Inputs
		a_r: in std_logic_vector(Nbit-2 downto 0);
		b_r: in std_logic_vector(Nbit-2 downto 0);
		cin_r: in std_logic;
		cout_r: out std_logic;
		
		-- Will store the result of a_r+b_r
		s_r: out std_logic_vector(Nbit-1 downto 0);
		clk: in std_logic;
		rst: in std_logic
	);
end Ripple_Carry_Adder_Pipelined;

architecture rtl of Ripple_Carry_Adder_Pipelined is 
	-- Building blocks of the Ripple Carry Adder Pipelined 
	component FULL_ADDER
		...
	end component FULL_ADDER;
	
	-- Need of a register to obtain the pipelined version
	component DFF
	...
	component DFF;
	
	-- Will propagate the carry signal among the whole architecture  
	signal carry_signal: std_logic_vector(Nbit-1 downto 1);
	
	-- Will store the outputs signal of the registers
	signal dff_signal: std_logic_vector(Nbit-1 downto 0) := (others => '0');

begin
	-- Implemented in a structured way in a similar fashion as seen in the Lab lessions
	GEN: for i in 1 to Nbit generate
		FIRST: if i=1 generate
		-- First FA
			FFI: FULL_ADDER port map (a_r(0), b_r(0), cin_r, s_r(0), carry_signal(1));
		end generate FIRST;
		INTERNAL: if i > 1 and i < Nbit generate
			-- Need of Register detection
			PIPE: if (i mod 3 = 0) generate
				DFF_I: DFF
					port map(
					d_dff      => carry_signal(i-1),
					clk_dff    => clk,
					resetn_dff => rst,
					q_dff      => dff_signal(i-1)
				);
				FFI: FULL_ADDER port map (a_r(i-1), b_r(i-1), dff_signal(i-1), s_r(i-1), carry_signal(i));
			end generate PIPE;
			
			-- No need of a register
			NOT_PIPE: if (i mod 3 /= 0) generate
				FFI: FULL_ADDER port map (a_r(i-1), b_r(i-1), carry_signal(i-1), s_r(i-1), carry_signal(i));
			end generate NOT_PIPE;           
		end generate INTERNAL;
		
		-- Implicit extension (the inputs have Nbit-2 bits, the output has Nbit-1 bits and there
		-- are Nbit-1 FA so the last bit is replicated in order to make the extension in the 
		-- correct way in C2 representation)
		
		LAST: if i=Nbit generate
		FFI: FULL_ADDER port map (a_r(Nbit-2), b_r(Nbit-2), carry_signal(Nbit-1), s_r(Nbit-1), cout_r);
		end generate LAST;
	end generate GEN;
end rtl;

\end{lstlisting}

\subsection{LUT}
This module will store every possible output of the sigmoid function in the input range of $[-11; +11]$ with 12 bits of precision. The input will be treated as an address signal to obtain the correct output value in the same fashion as shown in the Laboratory lectures.
\begin{lstlisting}[language=VHDL]
entity sigmoid_lut_2048 is
	port (
		address : in  std_logic_vector(10 downto 0);
		dds_out : out std_logic_vector(15 downto 0) 
	);
end sigmoid_lut_2048;


-- Output between [-11; +11], rapresented with fixed point 
-- [Reach the LSB method]
-- LSB(in) = (11)/(2^11 - 1) = 0.00537371763556424035173424523693
-- LSB(out) = (1)/(2^15 - 1) = 3.0518509475997192297128208258309e-5
-- inputs -> [-11, +11] 
-- outputs -> [0, 1]
-- Q(f(x)) = round(f(x)/LSB(out))*LSB(out)
--
-- What to store in the lut? round(f(x)/LSB(out)) for x in [0; 2047]*LSB(in)

architecture rtl of sigmoid_lut_2048 is
	type LUT_t is array (natural range 0 to 2047) of integer;
	constant LUT: LUT_t := (
		0 => 16384,
		1 => 16428,
		...
		2046 => 32766,
		2047 => 32766
	);

begin
	dds_out <= std_logic_vector(TO_SIGNED(LUT(TO_INTEGER(unsigned(address))),16));
end rtl;	
\end{lstlisting}

\subsubsection{Lut generation code}
The whole Lut was not compiled "by hand" obviously. The look-up table outputs were generated through the following python script by exploiting the computation concerning the LSB with 12 bits and 16 bits resolution made before.
\begin{lstlisting}[language=Python]
	import math
	
	#Calculate lsb of x (16 bits) and f(x) (12 bits)
	lsb_out = (1)/(2**15 - 1)
	lsb_in = (11)/(2**11 - 1)
	result = ""
	
	for x in range(0, 2048):
		f_x = (1)/(1 + math.exp(-(x*lsb_in)))
		lut = round(f_x/lsb_out)
		
		#Generate lut entries for every x
		result += str(x) + " => " + str(lut) + ",\n"
	
	print(result)
	
\end{lstlisting}