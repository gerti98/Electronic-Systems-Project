\section{VHDL CODE}
In this chapter will be presented the main modules that compose the architecture of the \textbf{Perceptron with sigmoid activation function}.
\subsection{Modules List}
As presented in the last chapter, I have followed a similar approach for creating the architecture. The following modules were created:

\begin{itemize}
	\item Perceptron
	\begin{itemize}
		\item Parallel$\_$Multiplier
  		\begin{itemize}
  			\item Unsigned Parallel Multiplier
  			\begin{itemize}
	  			\item Full Adder
	  			\item Half Adder
  			\end{itemize}
		\end{itemize}
	\end{itemize}
	\begin{itemize}
		\item Tree$\_$Adder
		\begin{itemize}
			\item Ripple$\_$Carry$\_$Adder$\_$Pipelined
			\begin{itemize}
				\item DFF
				\item Full Adder
			\end{itemize} 
		\end{itemize} 
	\end{itemize}
	\begin{itemize}
		\item Sigmoid$\_$Lut$\_$2048
	\end{itemize}
\end{itemize}

A \textbf{bottom-up strategy} was followed in order to build the architecture: starting from the some modules that will made up the architecture and after finishing each of them some testbenches were written in order to test each building block of the \textbf{Perceptron} (See next chapter for details).
\subsection{Perceptron}
The main hardware description of the architecture. In order to not show too much lines of code only the entity definition of this module will be shown.
\begin{lstlisting}[language=VHDL]
	library IEEE;
	use IEEE.std_logic_1164.all;
	use IEEE.numeric_std.all;
	entity Perceptron is
	port(
	
	-- x_1 to x_10 inputs of the perceptron with 8 bits
	x_1: in std_logic_vector(7 downto 0);
	x_2: in std_logic_vector(7 downto 0);
	x_3: in std_logic_vector(7 downto 0);
	x_4: in std_logic_vector(7 downto 0);
	x_5: in std_logic_vector(7 downto 0);
	x_6: in std_logic_vector(7 downto 0);
	x_7: in std_logic_vector(7 downto 0);
	x_8: in std_logic_vector(7 downto 0);
	x_9: in std_logic_vector(7 downto 0);
	x_10: in std_logic_vector(7 downto 0);
	
	-- w_1 to w_10 inputs of the perceptron with 9 bits
	w_1: in std_logic_vector(8 downto 0);
	w_2: in std_logic_vector(8 downto 0);
	w_3: in std_logic_vector(8 downto 0);
	w_4: in std_logic_vector(8 downto 0);
	w_5: in std_logic_vector(8 downto 0);
	w_6: in std_logic_vector(8 downto 0);
	w_7: in std_logic_vector(8 downto 0);
	w_8: in std_logic_vector(8 downto 0);
	w_9: in std_logic_vector(8 downto 0);
	w_10: in std_logic_vector(8 downto 0);
	
	-- b input of the perceptron with 9 bits
	b: in std_logic_vector(8 downto 0);
	
	clk: in std_logic;
	rst: in std_logic;
	
	-- output of the perceptron 16 bits
	f_z: out std_logic_vector(15 downto 0)
	);
	end Perceptron;
\end{lstlisting}

In the rest of this modules are instantiated and linked the various submodule that made up the \textbf{Perceptron} module.

\subsection{Parallel Multiplier}
\begin{lstlisting}[language=VHDL]
	library IEEE;
	use IEEE.std_logic_1164.all;
	use ieee.numeric_std.all;
	
	
	entity Parallel_Multiplier is
	generic (
	Nbit_a : positive; 
	Nbit_b: positive
	);
	port(
	a_p_signed: in std_logic_vector(Nbit_a - 1 downto 0);
	b_p_signed: in std_logic_vector(Nbit_b - 1 downto 0);
	p_signed: out std_logic_vector(Nbit_a + Nbit_b - 1 downto 0)
	);
	end entity Parallel_Multiplier;
	
	architecture rtl of Parallel_Multiplier is
	
	-- Building blocks of the Parallel Multiplier
	component Unsigned_Parallel_Multiplier
	generic(
	Nbit_a : positive;
	Nbit_b : positive
	);
	port(
	a_p: in  std_logic_vector(Nbit_a - 1 downto 0);
	b_p : in  std_logic_vector(Nbit_b - 1 downto 0);
	p   : out std_logic_vector(Nbit_a + Nbit_b - 1 downto 0)
	);
	end component Unsigned_Parallel_Multiplier;
	
	
	-- Unsigned component (will work for the unsigned parallel multiplier
	signal p_unsigned: std_logic_vector(Nbit_a + Nbit_b - 1 downto 0);
	signal a_p_unsigned: std_logic_vector(Nbit_a - 1 downto 0);
	signal b_p_unsigned: std_logic_vector(Nbit_b - 1 downto 0);
	
	-- will carry the sign bit for the signed rapresentation of the inputs
	signal a_sign: std_logic;
	signal b_sign: std_logic;
	
	begin
	
	-- Compute the unsigned representation from the signed one
	a_p_unsigned <= std_logic_vector(abs(signed(a_p_signed)));
	b_p_unsigned <= std_logic_vector(abs(signed(b_p_signed)));
	
	-- 2's complement rapresentation, the result sign uis computed through the xor op. between a and b
	p_signed <= std_logic_vector(unsigned(not(p_unsigned)) + 1) when (((a_sign xor b_sign) = '1')) else p_unsigned;
	
	-- Getting of the sign from a and b (the MSB of the C2 representation)
	a_sign <= a_p_signed(Nbit_a - 1);
	b_sign <= b_p_signed(Nbit_b - 1);
	
	unsigned_parallel_mul: Unsigned_Parallel_Multiplier
	generic map(
	Nbit_a => Nbit_a,
	Nbit_b => Nbit_b
	)
	port map(
	a_p =>  a_p_unsigned,
	b_p =>  b_p_unsigned,
	p   => p_unsigned
	);
	
	end architecture rtl; 
\end{lstlisting}

\subsection{Unsigned Parallel Multiplier}
\begin{lstlisting}[language=VHDL]
	library IEEE;
	use IEEE.std_logic_1164.all;
	
	entity Unsigned_Parallel_Multiplier is
	generic (
	Nbit_a : positive; 
	Nbit_b: positive
	);
	port(
	-- Unsigned representation of inputs
	a_p: in std_logic_vector(Nbit_a - 1 downto 0);
	b_p: in std_logic_vector(Nbit_b - 1 downto 0);
	
	-- p = a_p * b_p
	p: out std_logic_vector(Nbit_a + Nbit_b - 1 downto 0)
	);
	end entity Unsigned_Parallel_Multiplier;
	
	architecture rtl of Unsigned_Parallel_Multiplier is
	-- Building blocks of the Unsigned Parallel Multiplier
	component FULL_ADDER is
	port
	(
	a    : IN std_logic ;
	b    : IN std_logic ;
	cin  : IN std_logic ;
	s    : OUT std_logic ;
	cout : OUT std_logic 
	);
	end component;
	
	component HALF_ADDER is
	port
	(
	a    : IN std_logic ;
	b    : IN std_logic ;
	s    : OUT std_logic ;
	cout : OUT std_logic 
	);
	end component;
	
	-- Will hold the carry signals among the whole architecture
	signal carry_signal: std_logic_vector((Nbit_a - 1)*(Nbit_b - 1) - 1 downto 0);
	signal last_carry_signal: std_logic_vector((Nbit_b - 1) downto 0);
	
	-- Will hold the sum result of the FA and HA among the whole architecture
	signal sum_signal: std_logic_vector((Nbit_a - 1)*(Nbit_b - 2) - 1 downto 0);  
	
	-- will hold the precomputed values for the inputs a and b of the various Half Adder and Full Adder
	signal a_multiplier: std_logic_vector(Nbit_a + Nbit_b - 2 downto 0);
	signal b_multiplier: std_logic_vector((Nbit_a - 1)*(Nbit_b - 1) - 1 downto 0);
	
	begin
	
	-- First bit of the result
	p(0) <= (a_p(0) and b_p(0));
	
	
	-- Computation of the various inputs of each HA and FA
	d_process: process(a_p, b_p)
	begin
	
	for j in 1 to Nbit_b loop
	a_multiplier(j - 1) <= (a_p(0) and b_p(Nbit_b - j));
	end loop;
	
	for i in 2 to Nbit_a loop
	a_multiplier(Nbit_b + i - 2) <= (a_p(i - 1) and b_p(Nbit_b - 1));
	end loop;
	
	for i in 1 to Nbit_a-1 loop
	for j in 1 to Nbit_b - 1 loop
	b_multiplier((i-1)*(Nbit_b -1) + j - 1) <= (a_p(i) and b_p(Nbit_b - j - 1));
	end loop;
	end loop;
	end process d_process;
	
	
	-- Architecture will follow schema of the Parallel Multiplier
	-- Row index i
	GEN_a: for i in 1 to Nbit_a generate
	-- Column index j
	GEN_b: for j in 1 to Nbit_b - 1 generate 
	FIRST_ROW: if i=1 generate
	-- In the first Row only HA
	LEFT: if j < Nbit_b -1 generate
	ROW1_LEFT: HALF_ADDER
	port map
	(
	a    => a_multiplier(j - 1), 
	b    => b_multiplier(j - 1), 
	s    => sum_signal(j - 1), 
	cout => carry_signal(j - 1) 
	);
	end generate LEFT;
	RIGHT: if j = Nbit_b - 1 generate
	ROW1_RIGHT: HALF_ADDER
	port map
	(
	a    => a_multiplier(j - 1), 
	b    => b_multiplier(j - 1), 
	s    => p(1), -- Result bit 
	cout => carry_signal(j - 1) 
	);
	end generate RIGHT; 
	end generate FIRST_ROW;
	
	
	INTERNAL_ROW: if i > 1 and i < Nbit_a generate
	-- Internal Rows only FA
	LEFT: if j = 1 generate 
	ROW_INT_LEFT: FULL_ADDER
	port map
	(
	a => a_multiplier(Nbit_b + i - 2),  
	b => b_multiplier((i-1)*(Nbit_b -1) + j - 1), 
	cin => carry_signal((i-2)*(Nbit_b - 1) + (j-1)), 
	s => sum_signal((i-1)*(Nbit_b - 2) + (j-1)), 
	cout => carry_signal((i-1)*(Nbit_b - 1) + (j-1)) 
	);
	end generate LEFT;
	CENTER: if j > 1 and j < Nbit_b - 1 generate
	ROW_INT_CENTER: FULL_ADDER
	port map
	(
	a => sum_signal((i-2)*(Nbit_b - 2) + (j-2)), 
	b =>  b_multiplier((i-1)*(Nbit_b -1) + j - 1), 
	cin => carry_signal((i-2)*(Nbit_b - 1) + (j-1)),
	s => sum_signal((i-1)*(Nbit_b - 2) + (j-1)),
	cout => carry_signal((i-1)*(Nbit_b - 1) + (j-1))
	);
	end generate CENTER;
	RIGHT: if j = Nbit_b - 1 generate
	ROW_INT_RIGHT: FULL_ADDER
	port map
	(
	a => sum_signal((i-2)*(Nbit_b - 2) + (j-2)), 
	b => b_multiplier((i-1)*(Nbit_b -1) + j - 1), 
	cin => carry_signal((i-2)*(Nbit_b - 1) + (j-1)), 
	s => p(i), -- Result bit
	cout => carry_signal((i-1)*(Nbit_b - 1) + (j-1)) 
	);
	end generate RIGHT;
	end generate INTERNAL_ROW;
	
	
	LAST_ROW: if i = Nbit_a generate
	-- Last row FA and an HA on the rightmost block
	LEFT: if j = 1 generate 
	ROW_INT_LEFT: FULL_ADDER
	port map
	(
	a => a_multiplier(Nbit_b + i - 2), 
	b => carry_signal((i-2)*(Nbit_b - 1) + (j-1)), 
	cin => last_carry_signal(Nbit_b - j), 
	s => p((Nbit_a) + (Nbit_b) - 1 - j), -- Result bit
	cout => p((Nbit_a) + (Nbit_b) -1) -- Result bit
	);
	end generate LEFT;
	CENTER: if j > 1 and j < Nbit_b - 1 generate
	ROW_INT_CENTER: FULL_ADDER
	port map
	(
	a => sum_signal((i-2)*(Nbit_b - 2) + (j-2)), 
	b =>  carry_signal((i-2)*(Nbit_b - 1) + (j-1)), 
	cin => last_carry_signal(Nbit_b - j), 
	s =>  p((Nbit_a) + (Nbit_b) - 1 - j), -- Result bit
	cout =>last_carry_signal(Nbit_b - j + 1) 
	);
	end generate CENTER;
	RIGHT: if j = Nbit_b - 1 generate
	ROW_INT_RIGHT: HALF_ADDER
	port map
	(
	a => sum_signal((i-2)*(Nbit_b - 2) + (j-2)),
	b => carry_signal((i-2)*(Nbit_b - 1) + (j-1)), 
	s => p((Nbit_a) + (Nbit_b) -1 - j), -- Result bit
	cout =>last_carry_signal(Nbit_b - j + 1)
	);
	end generate RIGHT;
	end generate LAST_ROW;
	end generate GEN_b;
	end generate GEN_a;
	end architecture rtl; 
\end{lstlisting}

\subsection{Tree Adder}
\begin{lstlisting}[language=VHDL]
library IEEE;
use IEEE.std_logic_1164.all;

entity Tree_Adder is
port(
-- Inputs: result of the multiplication of xi*wi
in_1: in std_logic_vector(16 downto 0);
in_2: in std_logic_vector(16 downto 0);
in_3: in std_logic_vector(16 downto 0);
in_4: in std_logic_vector(16 downto 0);
in_5: in std_logic_vector(16 downto 0);
in_6: in std_logic_vector(16 downto 0);
in_7: in std_logic_vector(16 downto 0);
in_8: in std_logic_vector(16 downto 0);
in_9: in std_logic_vector(16 downto 0);
in_10: in std_logic_vector(16 downto 0);

-- Bias input
b: in std_logic_vector(8 downto 0);
clk: in std_logic;
rst: in std_logic;

-- Output
z: out std_logic_vector(20 downto 0)
);
end Tree_Adder;

\end{lstlisting}

\subsubsection{Ripple Carry Adder Pipelined}
\begin{lstlisting}[language=VHDL]
library IEEE;
use IEEE.std_logic_1164.all;

-- Realize a Ripple Carry Adder in a structural way

entity Ripple_Carry_Adder_Pipelined is
generic (Nbit: positive);
port(
-- Inputs
a_r: in std_logic_vector(Nbit-2 downto 0);
b_r: in std_logic_vector(Nbit-2 downto 0);
cin_r: in std_logic;
cout_r: out std_logic;

-- Will store the result of a_r+b_r
s_r: out std_logic_vector(Nbit-1 downto 0);
clk: in std_logic;
rst: in std_logic
);
end Ripple_Carry_Adder_Pipelined;

architecture rtl of Ripple_Carry_Adder_Pipelined is 
-- Building blocks of the Ripple Carry Adder Pipelined 
component FULL_ADDER
port(
a: in std_logic;
b: in std_logic;
cin: in std_logic;
s: out std_logic;
cout: out std_logic
);
end component FULL_ADDER;

-- Need of a register to obtain the pipelined version
component DFF
port(
d_dff      : in  std_logic;
clk_dff    : in  std_logic;
resetn_dff : in  std_logic;
q_dff      : out std_logic
);
end component DFF;

signal carry_signal: std_logic_vector(Nbit-1 downto 1);
signal dff_signal: std_logic_vector(Nbit-1 downto 0) := (others => '0');

begin
-- Implemented in a structured way in a similar fashion as seen in the Lab lessions
GEN: for i in 1 to Nbit generate
FIRST: if i=1 generate
-- First FA
FFI: FULL_ADDER port map (a_r(0), b_r(0), cin_r, s_r(0), carry_signal(1));
end generate FIRST;
INTERNAL: if i > 1 and i < Nbit generate
-- Need of Register detection
PIPE: if (i mod 3 = 0) generate
DFF_I: DFF
port map(
d_dff      => carry_signal(i-1),
clk_dff    => clk,
resetn_dff => rst,
q_dff      => dff_signal(i-1)
);
FFI: FULL_ADDER port map (a_r(i-1), b_r(i-1), dff_signal(i-1), s_r(i-1), carry_signal(i));
end generate PIPE;
-- No need of a register
NOT_PIPE: if (i mod 3 /= 0) generate
FFI: FULL_ADDER port map (a_r(i-1), b_r(i-1), carry_signal(i-1), s_r(i-1), carry_signal(i));
end generate NOT_PIPE;           
end generate INTERNAL;

-- Implicit extension (the inputs have Nbit-2 bits, the output has Nbit-1 bits and there
-- are Nbit-1 FA so the last bit is replicated in order to make the extension in the 
-- correct way in C2 representation)

LAST: if i=Nbit generate
FFI: FULL_ADDER port map (a_r(Nbit-2), b_r(Nbit-2), carry_signal(Nbit-1), s_r(Nbit-1), cout_r);
end generate LAST;
end generate GEN;
end rtl;

\end{lstlisting}

\subsection{LUT}
\begin{lstlisting}[language=VHDL]
library IEEE;
use IEEE.std_logic_1164.all;
use IEEE.numeric_std.all;

entity sigmoid_lut_2048 is
port (
address : in  std_logic_vector(10 downto 0);
dds_out : out std_logic_vector(15 downto 0) 
);
end sigmoid_lut_2048;


-- Output between [-11; +11], rapresented with fixed point
-- Need for 1 bit for integer, last 15 bit for float rapresentation
-- Reach the LSB method
-- 
-- LSB(in) = (11)/(2^11 - 1) = 0.00537371763556424035173424523693
-- LSB(out) = (1)/(2^15 - 1) = 3.0518509475997192297128208258309e-5
-- inputs -> [-11, +11] 
-- outputs -> [0, 1]
-- Q(f(x)) = round(f(x)/LSB(out))*LSB(out)
-- 
--
-- What to store in the lut? round(f(x)/LSB(out)) for x in [0; 2047]*LSB(in)

architecture rtl of sigmoid_lut_2048 is
type LUT_t is array (natural range 0 to 2047) of integer;
constant LUT: LUT_t := (
0 => 16384,
1 => 16428,
...
2046 => 32766,
2047 => 32766
);

begin
dds_out <= std_logic_vector(TO_SIGNED(LUT(TO_INTEGER(unsigned(address))),16));
end rtl;	
\end{lstlisting}

\subsubsection{Lut generation code}
\begin{lstlisting}[language=Python]
	"""
	LSB(out) = (1)/(2^15 - 1) = 3.0518509475997192297128208258309e-5
	LSB(in) = (11)/(2^11 - 1) = 0.00537371763556424035173424523693
	
	What to store in the lut? round(f(x)/LSB(out)) for x in [0; 2047]*LSB(in)
	"""
	
	import math
	
	#Calculate lsb of x (16 bits) and f(x) (12 bits)
	lsb_out = (1)/(2**15 - 1)
	lsb_in = (11)/(2**11 - 1)
	result = ""
	
	
	for x in range(0, 2048):
		f_x = (1)/(1 + math.exp(-(x*lsb_in)))
		lut = round(f_x/lsb_out)
		
		#Generate lut entries for every x
		result += str(x) + " => " + str(lut) + ",\n"
	
	print(result)
	
\end{lstlisting}