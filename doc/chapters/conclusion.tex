\section{Conclusion}
After performing the Synthesis with Xilinx VIVAVO, we can say that in order to head into the \textbf{implementation} and generating the bitstream of the \textbf{Perceptron} we could use \textbf{another board with an higher i/o capacity}. The implementation part was not executed because of the results obtained will be biased by the i/o planning constraint; in our case this would lead only to partial conclusion.\\
On the other hand, for improving more the clock performance of the architecture, we can increase the maximum frequency of the board by \textbf{adding pipeline registers in a more frequent manner} (in the implementation there was a register after 3 FA modules) with the drawback of an higher resource utilization and an higher number of clock cycles needed to get the result.\\
Another optimization could be done in the adder architecture by adding some \textbf{carry generation logic} in order to get results with a lower number of clock cycles, even if it will make the design phase more complex.